\section{Notes}

Why is situating a robot onto the human body an interesting and worthwhile project to explore?
- Passive vs active agent problem
    - Smart prosthetic are effectively passive agents mapping human input directly to an actuation state.
    - A social robot is an active agent separate from the self. We ascribe it motive and observe its actuation. 
    - Wearable robots could blend the line between these switching between active and passive robot states
- Robots situated on the human body live near-exclusively in the intimate social zone while in operation.
    - As such we expect interesting social breakdowns (if we cast this an active agent problem)
        - When should the robot engage in an activity? 
    - Do we gender the robot (and what are the users preferences there). Do we explore the sexual orientation of the robotic finger (if again treated as a multi-agent problem)
- Situated on the body offers access to the life of the user as they perform daily tasks
    - More importantly it affords affecting state in addition to gathering it
    - Cell-phones and wearables like smart bands/watches and eye wear approach this. It resolves the sensing (and in a limited case actuation ~ Holograms).
- How do these devices effect lifestyle e.g., clothing choice (style-design and accommodations). 
    - Would users be interested in making these modifications for an external agent
- Future is trans-human?

********* Human-robot proxemics: physical and psychological distancing in human-robot interaction - https://dl.acm.org/doi/abs/10.1145/1957656.1957786?casa_token=NzLUGM9j5h8AAAAA:OgyyX0LcWIy8Vx5KCcKLtkW3JjOt3C7_zszWkZTqGxJMcpLtYEqGbJ-Dn19KuRXPzbs4EkE5WRQa


Persuasive Robotics: The influence of robot gender on human behavior - https://ieeexplore.ieee.org/document/5354116

Why Not Marry a Robot? - https://link.springer.com/chapter/10.1007/978-3-319-57738-8_1

Gendering Humanoid Robots: Robo-Sexism in Japan - https://journals.sagepub.com/doi/abs/10.1177/1357034x10364767?casa_token=YzA2wWFYhbUAAAAA%3ALlXjMHKsPrYgK62k606Zyezta2BzICnnp9igc0O5rdOU2vHhqRwp7TSA3wS6jeBh-uTv58TLaoiH&

Can you Tell the Robot by the Voice? An Exploratory Study on the Role of Voice in the Perception of Robots - https://ieeexplore.ieee.org/abstract/document/8673305?casa_token=JhNYAIBBi00AAAAA:LNe96-KRXuqFraDABq9QUkjccU4jaqqrJi-MWyqZRzHqEb0wjNbfA9xlaTfHSFpiI5XT4I0S7g

Gendered robot voices influencing trust -
towards a robot recommendation system
 - https://webapps.cs.umu.se/uminf/reports/2020/001/part1.pdf#page=7


‘Robot Kung fu’: Gender and professional identity in biology and philosophy reviews - https://www.sciencedirect.com/science/article/abs/pii/S0378216607000161


Gendered voice and robot entities: Perceptions and reactions of male and female subjects - https://ieeexplore.ieee.org/abstract/document/5354204?casa_token=LeWFcZzeXMIAAAAA:Vg4rn1jN-5Tk2-GczNHKZfoEfE3HQZY2-GIxslEziaqMokZiBtIMq9GmXDg9Qe9GrUSlZBiROA

Is Your Roomba Male or Female? The Role of Gender Stereotypes and Cultural Norms in Robot Design - https://ojs.stanford.edu/ojs/index.php/intersect/article/view/171

The Philosophical Case for Robot Friendship - https://www.jstor.org/stable/10.5325/jpoststud.3.1.0005

When stereotypes meet robots: The double-edge sword of robot gender and personality in human–robot interaction - https://www.sciencedirect.com/science/article/pii/S0747563214002921?casa_token=GgEHIdrkIaUAAAAA:ORZsAA08cmv_2IoQh6Y6sm16wW8b-DQSqti4kJGuRGfSjGxPNmhLay5fL-yNtE3aKF-_jKji5w










Why supernumarary robotic fingers? What benefits to society does this confer? 























How do we gamify the pet finger process - (and why)? 
- Voluntary compliance / building habits (wearable fingers are not a well understood phenomena for the average wearer)
- Long-term study potential


Does Gamification Work? -- A Literature Review of Empirical Studies on Gamification - https://ieeexplore.ieee.org/abstract/document/6758978?casa_token=a1v-dQOLbWkAAAAA:tuz_-tb3hTz6mUQIPwBYBwevmB8pFvxuJ37bbtRSnWqR2HViaCcalcK1cn3HnvtYSMjEs1zTIQ



Tamagotchis need not die — Verification of statemate designs - https://link.springer.com/chapter/10.1007/BFb0054174


Virtual Pet Game Application to Train and Manage How to Raise a Cat Well and Properly - https://ieeexplore.ieee.org/abstract/document/9339682?casa_token=w9qx62Litl4AAAAA:qb-Qt3qJKFcCAiCRfsxnHcTdrRB-QRHdVQ0_hBl_Cpy72UhgkQPoUx5kV-4bB9Xiod9Cl7HN8w

Disposable Love: The Rise and Fall of a Virtual Pet - https://journals.sagepub.com/doi/abs/10.1177/14614449922225591?casa_token=Rg7xz-N_SJgAAAAA:sVW3Ju39hOwXeg621op6iE-Yz_YrAYt3RbHkNeZw4AxN0JQ_iPyjY_H-SYQSqjzv9kamUdrzh43x

3D Virtual Pet Multiplayer Game Dengan Menggunakan Google Cardboard - https://repository.its.ac.id/75819/

Virtual Pet: Trends of Development - https://link.springer.com/chapter/10.1007/978-3-030-37737-3_47

Bouncy: An Interactive Life-Like Pet - https://citeseerx.ist.psu.edu/viewdoc/download?doi=10.1.1.24.6293&rep=rep1&type=pdf

Design and implementation of an educational virtual pet using the OCC theory - https://link.springer.com/article/10.1007/s12652-011-0089-4

A study of integrating social networking service into the virtual pet web game system - https://ieeexplore.ieee.org/abstract/document/6765475

Developing a believable interactive agent using
virtual pet design - http://tombattey.com/wp-content/uploads/2019/09/Developing-a-believable-interactive-agent-using-virtual-pet-design.pdf

Pretty Pelvis: A Virtual Pet Application That Breaks Sedentary Time by Promoting Gestural Interaction - https://dl.acm.org/doi/abs/10.1145/2702613.2732807?casa_token=JkXQ_RldpdcAAAAA:WQkRpqz8am--wLzPUlMc7huRwsonOWnaJIxDXhA5UXoGN7ZfVrUVbo8Jci9uTnVK10bNDuuCe7qV

AIBO: Toward the Era of Digital Creatures - https://journals.sagepub.com/doi/abs/10.1177/02783640122068092?casa_token=EuzSS03wMncAAAAA:swLHVeZLNchoXfIhiFwRlTN1dpVQQjlkbqiNwWjgUQV1zvv9mMYIrwtM56wAK_hFifUYpTJ5wevr

Loving a virtual pet: Steps toward the technological erosion of emotion - https://www.proquest.com/docview/200590681?pq-origsite=gscholar&fromopenview=true




Methodology
A Self-Guiding Tool to Conduct Research With Embodiment Technologies Responsibly - https://internal-journal.frontiersin.org/articles/10.3389/frobt.2020.00022/full


self-experimentation / autobiographical


Revealing Tensions in Autobiographical Design in HCI - https://dl.acm.org/doi/abs/10.1145/3196709.3196781?casa_token=vh2oOPbiS8cAAAAA:rRxamZQ4p1ug3GD7g8qgyF1YJOai65nQFpbOPoWvEYxm3oxGr0xRaFwEpsZGNNj8_vbQhftJvx-O


Autobiographical design in HCI research: designing and learning through use-it-yourself - https://dl.acm.org/doi/abs/10.1145/2317956.2318034?casa_token=eLehAFz1mLYAAAAA:W1GVOp72FOw75gNOHEpFqo3mTqk2bt9qDg4LXzHgUrTwI900OwrzPo0uEz9-fhwMepTI5RswNlz9

A Sample of One: First-Person Research Methods in HCI - https://dl.acm.org/doi/abs/10.1145/3301019.3319996?casa_token=3wyiZhMB4JAAAAAA:2q49qOyThwM8-9geaIdixXoyOgYUv8GyIdwTx-1EIAT4YT7nP3VgzseQJyhiPzIKeOOslf-1nonN
