%%%%%%%%%%%%%%%%%%%%%%%%%%%%%%%%%%%%%%%%%%%%%%%%%%%%%%%%%%%%%%%%%%%%%%%%%%%%%%%%
%2345678901234567890123456789012345678901234567890123456789012345678901234567890
%        1         2         3         4         5         6         7         8

\documentclass[letterpaper, 10 pt, conference]{ieeeconf}  % Comment this line out if you need a4paper

%\documentclass[a4paper, 10pt, conference]{ieeeconf}      % Use this line for a4 paper

\IEEEoverridecommandlockouts                              % This command is only needed if 
                                                          % you want to use the \thanks command

\overrideIEEEmargins                                      % Needed to meet printer requirements.

%In case you encounter the following error:
%Error 1010 The PDF file may be corrupt (unable to open PDF file) OR
%Error 1000 An error occurred while parsing a contents stream. Unable to analyze the PDF file.
%This is a known problem with pdfLaTeX conversion filter. The file cannot be opened with acrobat reader
%Please use one of the alternatives below to circumvent this error by uncommenting one or the other
%\pdfobjcompresslevel=0
%\pdfminorversion=4

% See the \addtolength command later in the file to balance the column lengths
% on the last page of the document

% The following packages can be found on http:\\www.ctan.org
%\usepackage{graphics} % for pdf, bitmapped graphics files
%\usepackage{epsfig} % for postscript graphics files
%\usepackage{mathptmx} % assumes new font selection scheme installed
%\usepackage{times} % assumes new font selection scheme installed
%\usepackage{amsmath} % assumes amsmath package installed
%\usepackage{amssymb}  % assumes amsmath package installed



\begin{document}

\title{\LARGE \bf Exploration of a Wearable Supernumerary Finger as Embodied Robotic Companion: An Autobiographical Case Study}

\author{Curt Henrichs}

% My goal is to write a short (4-page) paper regarding this topic

\maketitle
\thispagestyle{empty}
\pagestyle{empty}


\begin{abstract}

\end{abstract}

\section{Introduction}

The relationship between an agent and its environment informs both form(s) and function(s) it takes. Agents able to successfully operate within their environment have a high "fitness". For robotics, designers often have to consider the degree of structure provided within the environment; the proximity and relationship with humans the robot may encounter; and the task/role the it ought to perform. In human-robot interaction scenarios, a "fit" robot is often one that can successfully navigate the vagaries of human behavior. 

Wearable robots pose an interesting space to explore this "fitness" design task. Most prominently, wearable robots exist within the intimate social zone of their wearer during use and thus are enmeshed within the lives of their wearer. For example, this affords the robot access to observe and act during social interactions between the wearer and third-party agents. The designer also needs to consider casting the wearable agent as independent versus an extension of the wearer. This choice informs the set of behaviors a robot ought to perform. A wearer will expect independent behavior from a wearable tele-presence robot, but less so a prosthetic. The environments that a wearable robot will find itself in is as unstructured as the human wearer's. This could pose challenges if the robot's goal requires recognizing and manipulating objects with an end-effector. Even the proper mounting of wearable robots on the wearer's body during manipulation poses challenges to the designer.

In this paper, we focus on a specific subset of wearable robotics used to augment hand functionality, called supernumerary robotic fingers\footnote{Part of the broader set of supernumerary robotic limbs.}. We built a prototype system--\textit{Taltosoid}--designed to act as an embodied, artificial pet; inspired by popularized virtual pets (e.g., Tamagotchis). This work aims to explore the design of such a system. We discuss our findings through several self-experimentation case studies culminating with a proposal on future directions necessary to develop a robust taxonomy of the design space.

Our contributions are:
\begin{itemize}
    \item Design and implementation of a prototype embodied, artificial pet supernumerary robotic finger (\textit{Taltosoid});
    \item Case studies exploring usability and functionality of \textit{Taltosoid} through self-experimentation.
\end{itemize}


\section{Background}

\subsection{Supernumerary Fingers}

\subsection{Robot Companions}

\subsection{Design Space}

% Introduce Taltosoid here

\section{Method \& Design}
To design a compelling interaction between the wearer and finger, we conducted a preliminary autobiographical evaluation. The purpose of this evaluation is to understand what the wearer's common activities are and capture initial breakdowns whilst wearing the hardware. We then designed an embodied companion system running on the hardware. We showcase the functionality of the companion with several formative case studies drawn from a day-long autobiographical session. See each subsection for further discussion.

\subsection{Day in the Life of a Wearer}
Our goal is to understand a wearer's common activities--specifically how they use their hands---throughout a typical day. Furthermore, we want to capture the long-term use breakdowns of the current hardware prototype. This necessitates intrusive entry into the life of the wearer. Hence, we opt for an autobiographical design approach were researchers are framed as the the target user; thus performing the evaluation. We discuss the literature and limitations of this approach in the discussion section. The researcher (we) performed a two-day evaluation, with day one dedicated to tracking typical hand use and day two focused on wearing a supernumerary robotic finger in a limited actuation mode. 

During the day, we wore a camera mounted on the left shoulder facing forward and down to capture a top-down view of the activities. We also mounted a camera on our lower arm to further capture hand activity. For day two we also wore the hardware in a limited actuation mode. In this mode, the device used two flex sensors mounted on the researchers index finger and thumb to guide thumb position using a simple heuristic. 

The evaluation day was considered started after we started recording. We started wearing the cameras as soon as we woke up. The day ended when we went to bed; stopping recording and taking off the cameras. During the day, when a camera's battery became critically low, we would stop recording and switch out batteries. We did not consider this part of ``daily-life activities'' and thus disregard it from coding and analysis.

All video data was then coded for hand activity themes and wearability breakdowns following a grounded theory approach. After the 

\textit{Day One - A typical day -}

\textit{Day Two - Wearing the prototype -}


% Purpose is to understand the weare's common activites and capture initial breakdowns


\subsection{Designing the Embodied Companion}


\subsection{Case Studies}
Following the same protocol as our first evaluation, we performed a day-long usability evaluation of the hardware + embodied companion. We wore cameras on both the shoulder and lower arm. We coded the video data for hand activity and wearability breakdowns. 

\textit{Day Three - Wearing the companion - }

\subsubsection{Case 1}

\subsubsection{Case 2}

\subsubsection{Case 3}




\section{Discussion}

% autobiographical research, the good/bad/and ugly


\end{document}